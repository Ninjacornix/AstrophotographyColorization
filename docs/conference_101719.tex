\documentclass[conference]{IEEEtran}
\IEEEoverridecommandlockouts
% The preceding line is only needed to identify funding in the first footnote. If that is unneeded, please comment it out.
\usepackage{cite}
\usepackage{amsmath,amssymb,amsfonts}
\usepackage{algorithmic}
\usepackage{graphicx}
\usepackage{textcomp}
\usepackage{xcolor}
\usepackage[T1]{fontenc}
\def\BibTeX{{\rm B\kern-.05em{\sc i\kern-.025em b}\kern-.08em
    T\kern-.1667em\lower.7ex\hbox{E}\kern-.125emX}}
\begin{document}

\title{Metode za automatsku
ravnotežu boja (auto white balance)
}

\author{\IEEEauthorblockN{1\textsuperscript{st} Lana Šprajc}
\IEEEauthorblockA{\textit{Fakultet eletrotehnike i računarstva} \\
\textit{lana.sprajc@fer.hr} }
\and
\IEEEauthorblockN{2\textsuperscript{nd} Given Name Surname}
\IEEEauthorblockA{\textit{dept. name of organization (of Aff.)} \\
\textit{name of organization (of Aff.)}\\
City, Country \\
email address or ORCID}
\and
\IEEEauthorblockN{3\textsuperscript{rd} Given Name Surname}
\IEEEauthorblockA{\textit{dept. name of organization (of Aff.)} \\
\textit{name of organization (of Aff.)}\\
City, Country \\
email address or ORCID}
\and
\IEEEauthorblockN{4\textsuperscript{th} Given Name Surname}
\IEEEauthorblockA{\textit{dept. name of organization (of Aff.)} \\
\textit{name of organization (of Aff.)}\\
City, Country \\
email address or ORCID}
\and
\IEEEauthorblockN{5\textsuperscript{th} Given Name Surname}
\IEEEauthorblockA{\textit{dept. name of organization (of Aff.)} \\
\textit{name of organization (of Aff.)}\\
City, Country \\
email address or ORCID}
}

\maketitle

\begin{abstract}
This document is a model and instructions for \LaTeX.
This and the IEEEtran.cls file define the components of your paper [title, text, heads, etc.]. *CRITICAL: Do Not Use Symbols, Special Characters, Footnotes, 
or Math in Paper Title or Abstract.
\end{abstract}

\begin{IEEEkeywords}
Hubble, astrofotografija, auto white, ravnoteža boja
\end{IEEEkeywords}

\section{Uvod}
Svemirski teleskop Hubble je program suradnje Europske svemirske agencije i
Nacionalne uprave za aeronautiku i svemir. Pojedinačne slike s Hubbleovih kamera
ne sadržavaju nikakvu informaciju o boji kao takvoj, osim boje filtra, koji
odabire niz valnih duljina iz cijelog spektra svjetlosti. Crno-bijela (jednobojna)
slika najrealističnije predstavlja raspon svjetline u takvoj jednoj slici. Usprkos
tomu, slike u boji mogu se rekonstruirati kombiniranjem nekoliko slika napravljenih
kroz različite filtre i dodjeljivanjem različite boje svakoj slici. Želimo testirati
metode balansiranja boja na astrofotografijama napravljenima koristeći
Svemirski teleskop Hubble.

\section{Pregled literature}
Vrijednost piksela ovisi o boji izvora, refleksiji površine i osjetljivosti kamere.
Budući da ovi parametri nisu poznati, potrebne su metode procjene parametara \cite{article}.

Jedna od skupina metoda procjene boje izvora su statičke metode. U njoj se primjenjuju
fiksni parametri nad ulaznom slikom. Najkorištenija statička metoda je pretpostavka sivog
svjeta (engl. \textit{Grey-World Assumption}). Početna pretpostavka ove metode jest da je
prosječna refleksija površine siva te se zaključuje da ukupna prosječna boja u slici proizlazi
iz izvora. Tako se pojedina RGB komponenta izvora računa kao integral te komponent kroz
cijelu sliku pomnožen s nekim skalarom (kako bi se izvor normirao) \cite{BUCHSBAUM19801}.

Druga skupina često korištenih metoda procjene boje su metode bazirane na gamutu.
Osnovna pretpostavka ove metode je da za dani izvor svjetlosti možemo primjetiti
samo ograničeni skup boja. U fazi treniranja ove metode određuje se koji skup boja
možemo očitati za određene izvore. Nakon treniranja, za ulaznu sliku odredi se skup
boja koje ona sadržava te se definira koji su mogući izvori svjetlosti za taj skup boja.
Nakon toga se odabire neki od mogućih izvora svjetlosti i primjenjuje se tranformacija nad
slikom kako bi se dobila balansirana slika \cite{Forsyth1990}.


\section{Opis rješenja}

\section{Opis rezultata}

\section{Diskusija}

\section{Zaključak}

\bibliographystyle{IEEEtran}
\bibliography{IEEEfull}
\vspace{12pt}

\end{document}
