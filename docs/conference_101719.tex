\documentclass[conference]{IEEEtran}
\IEEEoverridecommandlockouts
% The preceding line is only needed to identify funding in the first footnote. If that is unneeded, please comment it out.
\usepackage{cite}
\usepackage{amsmath,amssymb,amsfonts}
\usepackage{algorithmic}
\usepackage{graphicx}
\usepackage{textcomp}
\usepackage{xcolor}
\usepackage[T1]{fontenc}
\def\BibTeX{{\rm B\kern-.05em{\sc i\kern-.025em b}\kern-.08em
    T\kern-.1667em\lower.7ex\hbox{E}\kern-.125emX}}
\begin{document}

\title{Metode za automatsku
ravnotežu boja (auto white balance)
}

\author{\IEEEauthorblockN{1\textsuperscript{st} Lana Šprajc}
\IEEEauthorblockA{\textit{Fakultet eletrotehnike i računarstva} \\
\textit{lana.sprajc@fer.hr} }
\and
\IEEEauthorblockN{2\textsuperscript{nd} Given Name Surname}
\IEEEauthorblockA{\textit{dept. name of organization (of Aff.)} \\
\textit{name of organization (of Aff.)}\\
City, Country \\
email address or ORCID}
\and
\IEEEauthorblockN{3\textsuperscript{rd} Given Name Surname}
\IEEEauthorblockA{\textit{dept. name of organization (of Aff.)} \\
\textit{name of organization (of Aff.)}\\
City, Country \\
email address or ORCID}
\and
\IEEEauthorblockN{4\textsuperscript{th} Given Name Surname}
\IEEEauthorblockA{\textit{dept. name of organization (of Aff.)} \\
\textit{name of organization (of Aff.)}\\
City, Country \\
email address or ORCID}
\and
\IEEEauthorblockN{5\textsuperscript{th} Given Name Surname}
\IEEEauthorblockA{\textit{dept. name of organization (of Aff.)} \\
\textit{name of organization (of Aff.)}\\
City, Country \\
email address or ORCID}
}

\maketitle

\begin{abstract}
This document is a model and instructions for \LaTeX.
This and the IEEEtran.cls file define the components of your paper [title, text, heads, etc.]. *CRITICAL: Do Not Use Symbols, Special Characters, Footnotes, 
or Math in Paper Title or Abstract.
\end{abstract}

\begin{IEEEkeywords}
Hubble, astrofotografija, auto white, ravnoteža boja
\end{IEEEkeywords}

\section{Uvod}
Svemirski teleskop Hubble je program suradnje Europske svemirske agencije i
Nacionalne uprave za aeronautiku i svemir. Pojedinačne slike s Hubbleovih kamera
ne sadržavaju nikakvu informaciju o boji kao takvoj, osim boje filtra, koji
odabire niz valnih duljina iz cijelog spektra svjetlosti. Crno-bijela (jednobojna)
slika najrealističnije predstavlja raspon svjetline u takvoj jednoj slici. Usprkos
tomu, slike u boji mogu se rekonstruirati kombiniranjem nekoliko slika napravljenih
kroz različite filtre i dodjeljivanjem različite boje svakoj slici. Želimo testirati
metode balansiranja boja na astrofotografijama napravljenima koristeći
Svemirski teleskop Hubble.

\section{Pregled literature}
Vrijednost piksela ovisi o boji izvora, refleksiji površine i osjetljivosti kamere.
Budući da ovi parametri nisu poznati, potrebne su metode procjene parametara \cite{article}.

Jedna od skupina metoda procjene boje izvora su statičke metode. U njoj se primjenjuju
fiksni parametri nad ulaznom slikom. Najkorištenija statička metoda je pretpostavka sivog
svjeta (engl. \textit{Grey-World Assumption}). Početna pretpostavka ove metode jest da je
prosječna refleksija površine siva te se zaključuje da ukupna prosječna boja u slici proizlazi
iz izvora. Tako se pojedina RGB komponenta izvora računa kao integral te komponent kroz
cijelu sliku pomnožen s nekim skalarom (kako bi se izvor normirao) \cite{BUCHSBAUM19801}.

Druga skupina često korištenih metoda procjene boje su metode bazirane na gamutu.
Osnovna pretpostavka ove metode je da za dani izvor svjetlosti možemo primjetiti
samo ograničeni skup boja. U fazi treniranja ove metode određuje se koji skup boja
možemo očitati za određene izvore. Nakon treniranja, za ulaznu sliku odredi se skup
boja koje ona sadržava te se definira koji su mogući izvori svjetlosti za taj skup boja.
Nakon toga se odabire neki od mogućih izvora svjetlosti i primjenjuje se tranformacija nad
slikom kako bi se dobila balansirana slika \cite{Forsyth1990}.


\section{Opis rješenja}

Slike su dane u .fits formatu. Fits je format otvorenog standarda koji se koristi u prijenosu fotografija s teleskopa. Budući da se različiti tipovi podataka mogu prenositi u tom formatu, sama slika je formatirana kao 2D polje gdje svaki element polja predstavlja intenzitet točke. Promatramo 2 tipa fotografija (reprezentiranih kao 2D polje), a to su UVIS (Ultraviolet Imaging Spectrograph) i IR (Infrared) fotografije. Budući da slike ne nose informaciju o boji to je potrebno primijeniti. Naime teleskop slika na različitim valnim duljinama elemente kao što su kisik, vodik i sumpor. Valna duljina dvo dimenzionalnog polja je dana u samom opisu datoteke. Uz navedeno dane su i informacije o tipu korištenog senzora i informacija je li polje težina ili "drizzled image". Pri izradi koristimo biblioteku Astropy za Python3. Prvo ćemo učitati matrice iz fits slika, te osigurati da sve NaN vrijednosti stavimo na 0 (ove vrijednosti se pojavljuju zbog mikro oštećenja na senzoru i radijacije). No ne možemo još složiti našu sliku. Primijetimo da valne duljine koje smo dobili odgovaraju crvenoj boji, zbog čega? Prisjetimo se da pad frekvencije odgovara "red shiftu" jer se objekti koje snimamo udaljavaju od nas, no i dalje se vide razlike u valnim duljinama. Ono što nam ostaje je mapirati te valne duljine na crvenu, plavu i zelenu. Ako smo dobili 3 datoteke od kojih jedna odgovara 502nm (plava boja), te druge dvije 657nm (crvena boja) i 673nm (crvena boja). Ono što ćemo uraditi je to da ćemo uzeti sliku od 657nm i spustiti ju u područje ispod 600nm. No ni to ne daje najbolje rezultate jer smo umjetno mijenjali jednu valnu duljinu a da smo ostale valne duljine ostavili na svom mjestu. Iz tog razloga smo tražili 99 percentil svakog kanala te maksimalni intenzitet točke (koji je 256) podijelili s tim brojem. Time smo dobili omjere koje ćemo množiti s originalnim vrijednostima intenziteta točke te ih fiksirati u omjer od 0 do 256. Tim smo balansirali sve kanale matrice, no ni dalje ne izgleda jako lijepo. Ono što možemo pokušati je promijeniti krivulju boje. Uzet ćemo model linearne regresije naučen na svojoj boji, kojoj su ciljne vrijednosti ostale boje. Ona točna boja će odgovarati aritmetičkoj sredini npr. modela crvene boje s ciljnim vrijednostima zelene boje i modela crvene boje s ciljnim vrijednostima plave boje. Nakon što te vrijednosti ubacimo u interval od 0 do 256 dobivamo balansirane krivulje boja. Sljedeći korak je smanjiti šum slike, za to ćemo koristiti "Non-Local Means Denoising" algoritam koji je ugrađen u paket cv2. Na kraju ako želimo da slika bude oku ugodnija možemo promijeniti svjetlinu i živost slike. Prvo ćemo sliku pretvoriti iz rgb u hsv sustav. Nakon toga množiti v i s vrijednosti s konstantama koje se nama osobno više sviđaju (ova procedura je empirijska te se gube detalji). Što se tiče infracrvenih slika tu dobivamo 2 slike, obično jedna slika predstavlja plavi kanal dok druga predstavlja žutu ili narančastu boju, te je potrebno odvojiti taj kanal u 2 umjetna kanala, te ponoviti proceduru kao i za uvis slike.

\section{Opis rezultata}

\section{Diskusija}

\section{Zaključak}

\bibliographystyle{IEEEtran}
\bibliography{IEEEfull}
\vspace{12pt}

\end{document}
